%%
%% Copyright (c) 2019-2020 Qian Sun <sqandzxy@gmail.com>
%%
%%

% English version
\documentclass{resume}
\pagestyle{fancy}
\newcommand{\Matlab}{\textsc{Matlab}\textregistered }
\newcommand{\Ansys}{\textsc{Ansys}\textregistered }
\newcommand{\Mathematica}{\textsc{Mathematica}\textregistered }
\newcommand{\Microsoft}{\textsc{Microsoft}\textregistered }
\newcommand{\Solidworks}{\textsc{Solidworks}\textregistered }

%\linespread{1.1}

% File information shown at the footer of the last page

\name{Qian}{Sun}

\keywords{Matlab, Mathematica, Programming, Construction, Material, Modalanalyse}

% \tagline{\icon{\faBinoculars}} <position-to-look-for>}
% \tagline{<current-position>}

% \photo{<height>}{<filename>}

\profile{
  \mobile{015257541682}
  \email{sq43793911@outlook.com}
  \github{sq43793911} \\
  \university{Techinische Universität Bergakademie Freiberg}
  \degree{Master of Science}\\
  \birthday{21.09.1991}
  \home{Freiberg \textbullet Deutschland}
  % Custom information:
  % \icontext{<icon>}{<text>}
  % \iconlink{<icon>}{<link>}{<text>}
}

\begin{document}
\makeheader


%======================================================================
% Summary & Objectives
%======================================================================
%\begin{spacing}{1.4}
%	{\large Master in \textbf{Fahrzeugbau: Werkstoffe und Komponenten} mit guten Grundlagen in Physik, Mathematik und Materialwissenschaften. Gute Kenntnisse in Strukturmodellierung, \textbf{FEM} und Modalanalyse. Begeisterung in Computer- und Netzwerktechnologien, gute Kompetenzen in der Verwendung von Linux, SSH, Git und anderen Plattformen und Tools. Sehr gute Kenntnisse in wissenschaftlicher Computersoftware, z.B. \textbf{\Matlab} \ und \Mathematica, Erfahrung in FEM-Software wie \Ansys.}
%\end{spacing}

%======================================================================
\sectionTitle{Kompetenzen \& Sprache}{\faWrench}
%======================================================================
\begin{spacing}{1.4}
\begin{competences}[10em]

  \comptence{{\large Fachgebiet}}{
   {\large  \textbf{Elektromobilität}, \quad \textbf{FEM}, \quad \textbf{Modalanalyse}, \quad  \textbf{Leichtbau}, \quad MCU-System}
  }
  \comptence{{\large Programmieren}}{
    {\large \textbf{\Matlab}, \quad \Mathematica, \quad R, \quad Python\textregistered, \quad C}
  }
  \comptence{{\large Tools}}{
       {\large \textbf{\Ansys} Products, \quad \textbf{\Solidworks}, \quad Git, \quad \textbf{\LaTeX}, \quad \Microsoft \textbf{Office}, \quad Freescale\textregistered \textbf{CodeWarrior}, \quad IAR\textregistered Embedded Workbench}
  }
  \comptence{\icon{\faLanguage} {\large Sprache}}{
{\large   	\textbf{Deutsch} --- C1; \ 
    \textbf{English} --- B1;\ \ \textbf{Chinesisch} --- Muttersprache} 
  }
\end{competences}
\end{spacing}



%======================================================================
\sectionTitle{Bildung}{\faGraduationCap}
%======================================================================
\begin{spacing}{1.3}
\begin{educations}
	\education%
	{{\large 07.2014}}%
	[{\large 09.2010}]%
	{{\large Liaocheng University}}%
	{{\large School of Mechanical and Automotive Engineering}}%
	{\textbf{{\large Fahrzeugtechnik}}}%
	{{\large Bachelor}}
	
	\separator{1ex}
	
	\education%
	{{\large 07.2020}}%
	[{\large 10.2016}]%
	{{\large Technische Universität Bergakademie Freiberg}}%
	{{\large Fakultät 5 - Werkstoffwissenschaft und Werkstofftechnologie}}%
	{\textbf{{\large Fahrzeugbau: Werkstoffe und Komponenten}}}%
	{{\large Master}}
\end{educations}
\end{spacing}

%======================================================================
\sectionTitle{Kenntnisse}{\faCogs}
%======================================================================
\begin{itemize}
	\setlength{\itemsep}{4pt}
	
  \item {\large \textit{Fahrzeugtechnik}\hspace{1mm}:}\\[1mm]
	{\large \textbf{Fahrzeugleichtbau}, \hspace{1mm} \textbf{Elektromobilität}, \hspace{1mm} Fahrzeugkomponenten und Theorie.}
	
  \item {\large \textit{Werkstofftechnik}\hspace{1mm}:}\\[1mm]
	{\large \textbf{Hochtemperaturwerksstofftechnik}, \hspace{1mm} Werkstoffbearbeitung und Gießerei, \hspace{1mm} Bionik.}
	
  \item {\large \textit{Mechanik und Dynamik}\hspace{1mm}:} \\[1mm]
	{\large \textbf{Modalanalyse},\hspace{1mm} Strukturanalyse und Simulation,\hspace{1mm} Dynamikanalyse und Schwingungs\-simulation.}
	
  \item {\large \textit{Computerwissenschaften}\hspace{1mm}:} \\[1mm]
	{\large \textbf{Design und Debugging von MCU-System}, Design und Debugging von Regelungssystem.}
	
  \item {\large \textit{Anders}\hspace{1mm}:} \\[1mm]
	{\large \textbf{Optimierungsdesign},\hspace{1mm} Neuronales Netz und tiefes Lernen, \hspace{1mm} Big-Data-Analyse.}
\end{itemize}


%======================================================================
\sectionTitle{Personale Projekte}{\faCode}
%======================================================================
\begin{itemize}
	
	\begin{spacing}{1.4}
	\setlength{\itemsep}{12pt}
	\item 
	\link{https://github.com/sq43793911/Masterarbeit_public}{\texttt{\textbf{\underline{{\large Masterarbeit:}}}}} (\textit{Matlab\textregistered}, \textit{Mathematica\textregistered}, \textit{Ansys\textregistered}, \textit{Solidworks\textregistered}, \textit{\LaTeX}, \textit{FEM}, \textit{PULSE LabShop}, \textit{ME'scopeVES})\\[1mm]
		\link{https://github.com/sq43793911/Masterarbeit_public}{\githubSymbol $ \Rightarrow $ https://github.com/sq43793911/Masterarbeit\_public }\\
		\link{https://github.com/sq43793911/Masterarbeit_public}{\color{black} {\large \textbf{Experimentelle und simulative Modalanalyse eines Werkzeugschaftes beim HSC-Frä\-sen unter Einfluss eines Exzentrizitätsfehlers.} Im Rahmen der ausgeschriebenen Arbeit ist das dynamische Verhalten eines Werkzeugschaftes beim HSC-Fräsen zu untersuchen. Der Einfluss einer Unwucht (Exzentrizität) soll dabei im Vordergrund stehen. Neben der Simulation soll auch eine experimentelle Untersuchung erfolgen. Dazu ist eine Versuchsvorrichtung zu konstruieren, zu fertigen und in Betrieb zu nehmen. Die FE Simulation wird mit \Matlab \ durchgeführt. Außerdem wird das CAD Programm \Solidworks \ zur Darstellung der Bauteilen benutzt. In Bezug auf die Messsoftware werden PULSE LabShop und ME'scopeVES verwendet.}}
	
	\item \link{https://github.com/sq43793911/Projektarbeit_public}{\texttt{\textbf{\underline{{\large Projektarbeit:}}}}} (\textit{Matlab\textregistered, Mathematica\textregistered, Ansys\textregistered, \LaTeX, FEM})\\[1mm]
		\link{https://github.com/sq43793911/Projektarbeit_public}{\githubSymbol $ \Rightarrow $ https://github.com/sq43793911/Projektarbeit\_public }\\
		\link{https://github.com/sq43793911/Projektarbeit_public}{\color{black} {\large \textbf{Modalanalyse mit Hilfe der Finite-Elemente-Methode.} Im Rahmen der ausgeschriebenen Arbeit ist ein Programm zur numerischen Modalanalyse mittels der Finite-Elemente-Methode zu entwickeln. Die FE Simulation wird mit \Matlab durchgeführt. Außerdem werden die FE-Software \Ansys zum Vergleichen der Ergebnisse benutzt.}}

	  \item \link{https://github.com/sq43793911/Intelligent-Algorithms}{\texttt{\textbf{\underline{{\large Forschungsprojekt für intelligente Algorithmen:}}}}}(\textit{Matlab\textregistered})\\[1mm]
	  	\link{https://github.com/sq43793911/Intelligent-Algorithms}{\githubSymbol $ \Rightarrow $ https://github.com/sq43793911/Intelligent-Algorithms }\\
		\link{https://github.com/sq43793911/Intelligent-Algorithms}{\color{black} {\large $ \mathbf{a)}  $ Vier gängige intelligente Optimierungsalgorithmen (Genetischer-, Ameisenkolonial-, Immun-, Tabutabellenalgorithmus) werden verwendet, um das TSP zu analysieren, und die Vor- und Nachteile verschiedener Algorithmen und ihr Anwendungswert werden verglichen. \\[1mm]
		$ \mathbf{b)}  $ Der genetische Algorithmus wird verwendet, um die Analyse des Kostenmanagements für die Lagerung und Bestellung von Fabrik zu optimieren. \\[1mm]
		$ \mathbf{c)}  $ Basierend auf der Classifacation Learner Toolbox von Matlab werden kNN(k Nearest Neighbour Classification-Algorithmus)- und Cluster-Analyse-Algorithmen studiert und geübt, die üblicherweise im Data Mining verwendet werden.}}
	
  	\item \link{https://github.com/sq43793911/Balance_Car}{\texttt{\textbf{\underline{{\large Zweirad-Selbstbalance-Fahrzeug:}}}}}(\textit{Freescale\textregistered CodeWarrior, Freescale\textregistered S12X, C, CCD}) \\[1mm]
  		\link{https://github.com/sq43793911/Balance_Car}{\githubSymbol $ \Rightarrow $ https://github.com/sq43793911/Balance\_Car }\\
    	\link{https://github.com/sq43793911/Balance_Car}{\color{black} {\large Ein Zweirad-Selbstbalance-Fahrzeug, das auf der der Anforderungen des Freescale Smart-Car-Cup entwickelt und hergestellt wurde und über automatische Funktionen zur Spurerkennung und -steuerung verfügt. Das Steuerungssystem basiert auf dem Freescale S12X-Mikrocontroller und verwendet eine CCD-Kamera zur Spurerkennung. }}
\vspace{-5mm}
  	\item \link{https://github.com/sq43793911/Construction_Analysis}{\texttt{\textbf{\underline{{\large Analysieren der Schwingungssysteme:}}}}}(\textit{Matlab\textregistered})\\[1mm]
  		\link{https://github.com/sq43793911/Construction_Analysis}{\githubSymbol $ \Rightarrow $ https://github.com/sq43793911/Construction\_Analysis }\\
		\link{https://github.com/sq43793911/Construction_Analysis}{\color{black} {\large Ungedämpfte Schwingungen, gedämpfte Schwingungen, die Zeitbereichseigenschaften des Maxwell-Modells und die Reaktionseigenschaften nach der Anregung werden analy\-siert und die entsprechende Bilder werden erzeugt.}}
		  
	\item \link{https://github.com/sq43793911/resume}{\texttt{\textbf{\underline{{\large Lebenslauf:}}}}}(\textit{\LaTeX})\\[1mm]
		\link{https://github.com/sq43793911/resume}{\githubSymbol $ \Rightarrow $ https://github.com/sq43793911/resume }\\
		\link{https://github.com/sq43793911/resume}{\color{black} {\large Mit \LaTeX \ wird der gerade gelesene Lebenslauf erstellt, um die Grundfunktionen und die Verwendung von \LaTeX \ zu erlernen.}}
		
	
	\end{spacing}
\end{itemize}
\clearpage

\end{document}
