%%
%% Copyright (c) 2019-2020 Qian Sun <sqandzxy@gmail.com>
%%
%%

% Chinese version
\documentclass[zh]{resume}
\pagestyle{fancy}

\newcommand{\Matlab}{\textsc{Matlab}\textregistered }
\newcommand{\Ansys}{\textsc{Ansys}\textregistered }
\newcommand{\Mathematica}{\textsc{Mathematica}\textregistered }
\newcommand{\Microsoft}{\textsc{Microsoft}\textregistered }
\newcommand{\Solidworks}{\textsc{Solidworks}\textregistered }

\name{谦}{孙}

\keywords{Matlab, Mathematica, Programming, Construction, Material, Modalanalyse}

% \tagline{\icon{\faBinoculars}} <position-to-look-for>}
% \tagline{<current-position>}

% \photo{<height>}{<filename>}

\profile{
	\mobile{136-5168-6379}
	\email{sq43793911@outlook.com}
	\github{sq43793911} \\
	\university{弗赖贝格工业大学(Techinische Universität Bergakademie Freiberg)} \\
	\degree{理学 \textbullet 硕士}
	\birthday{1991-09-21}
	\home{中国 \textbullet 山东}
	% Custom information:
	% \icontext{<icon>}{<text>}
	% \iconlink{<icon>}{<link>}{<text>}
}


\begin{document}
	\makeheader
	
%	======================================================================
%	 Summary & Objectives
%	======================================================================
	%{\onehalfspacing\hspace{2em}%
{\large 	汽车制造专业研究生,有扎实的汽车技术、市场知识,擅长汽车技术的优劣分析、比较,热衷于研究汽车市场的技术动向、前景,喜欢从技术角度出发,对汽车市场的未来走向和发展进行预测。\\[2mm] 
    拥有长达4年多的德国留学经历,具有优秀的德语沟通交流能力,对德国的人文、经济、政治环境都有非常充分的认识,尤其是对汽车行业有深入的了解,在校期间同大众、宝马、奥迪等公司的客座教授进行过交流,对汽车行业有自己的观点和看法。\\[2mm] 
	同时在国外学习期间,参加过国内公司和德国公司的合作谈判,并担任随行的日常和专业技术的德语翻译。对德国的企业文化和商务环境有深入的了解。
	硕士在校成绩优秀,最终排名为前30\%,具有积极实践和科学探索精神,并积极参与多个项目和比赛。拥有非常优秀的计算机技术,熟练使用各种办公软件以及专业软件。}
% \par }
	
	%======================================================================
	\sectionTitle{技能和语言}{\faWrench}
	%======================================================================
	\begin{spacing}{1.6}
		\begin{competences}
			\comptence{专业领域}{%
				{\large \textbf{新能源汽车},\quad \textbf{汽车轻量化},\quad \textbf{汽车市场分析},\quad \textbf{汽车德语}}
			}
					
			\comptence{工具}{%
			{\large \Microsoft \textbf{Office},   \quad  \Microsoft \textbf{Visio}, \quad \textbf{Adobe Acrobat}, \quad Github, \quad \LaTeX, }}

		
			\comptence{编程}{%
				{\large \textbf{Matlab\textregistered}, \quad Mathematica\textregistered,  \quad Python\textregistered, \quad C++\textregistered } }
			
			\comptence{\icon{\faLanguage} 语言}{
				{\large \textbf{德语} --- 精通;  \quad  \textbf{英语} --- 熟练}
			}
		\end{competences}
	\end{spacing}
	
	%======================================================================
	\sectionTitle{教育背景}{\faGraduationCap}
	%======================================================================
	\begin{spacing}{1.6}
		\begin{educations}
			
			\education%
			{{\large 2014.07}}%
			[{\large 2010.09}]%
			{{\large 聊城大学}}%
			{{\large 机械与汽车工程学院}}%
			{{\large 车辆工程}}%
			{{\large \textbf{学士}}}
			
			\separator{0.5ex}
			
			\education%
			{{\large 2020.07}}%
			[ {\large 2016.10} ]%
			{ {\large 弗赖贝格工业大学(Technische Universität Bergakademie Freiberg)} }%
			{ {\large 材料科学与技术系(Fakultät 5-Werkstoffwissenschaft und Werkstofftechnologie)} }%
			{ {\large 汽车制造(材料与部件)(Fahrzeugbau: Werkstoffe und Komponenten)} }%
			{{\large \textbf{硕士}}}
			
		\end{educations}
	\end{spacing}


%======================================================================
\sectionTitle{实习经历}{\faBriefcase}
%======================================================================
\begin{spacing}{1.6}
\begin{experiences}
	\experience%
	[{\large 2016.08}]%
	{{\large 2015.09}}%
	{{\large \textbf{售后服务} @ 山东鲁沪汽车销售有限公司}}%
	[\begin{itemize}
		\item {\large 负责车辆售后信息的处理和技术分析,熟练使用Word,Excel等办公软件。}
	\end{itemize}]
\end{experiences}
	\end{spacing}
%======================================================================
% Awards / Scholarships / Certificates
\sectionTitle{获奖及证书}{\faHeart}
%======================================================================
\begin{spacing}{1.6}
\begin{entries}
	\entry{{\large 2013.06}}%
	{{\large 第一届山东省大学生汽车知识竞赛 \textbullet 第四名}}
	\entry{{\large 2013.08}}%
	{{\large 第八届飞思卡尔智能车大赛山东赛区 \textbullet 优胜奖} }
	\entry{{\large 2015.11}}%
	{{\large Test-Daf 德福语言证书 \textbullet 16分(满分20})}
	\entry{{\large 2016.03}}%
	{{\large APS留德人员审核部 \textbullet 学历审核证书}}
%	\entry{2011.09}%
%	{全国计算机等级考试 \textbullet 四级网络工程师}
\end{entries}
		\end{spacing}
%	%======================================================================
%	\sectionTitle{专业技能}{\faCogs}
%	%======================================================================
%	\begin{itemize}
%		\setlength{\itemsep}{5pt}
%		\item \textit{\underline{{\large 车辆工程}}}:\quad {\large \textbf{汽车技术市场分析},\quad \textbf{新能源汽车},\quad 汽车构造和理论}
%		\item \textit{\underline{{\large 材料科学}}}: \quad {\large \textbf{金属高温热处理技术},\quad \textbf{摩擦和磨损分析},\quad 金属腐蚀分析和腐蚀防护,仿生材料}
%		\item \textit{\underline{{\large 力学和运动学}}}: \quad {\large \textbf{模态分析},\quad 结构分析和仿真,振动系统分析和仿真}
%		\item \textit{\underline{{\large 计算机学}}}: \quad {\large \textbf{MCU系统的设计和调试}, \quad 控制系统的设计和调试}
%		\item \textit{\underline{{\large 其他}}}: \quad {\large \textbf{最优化设计},\quad 神经网络,大数据分析}
%	\end{itemize}
%	
	%%======================================================================
	%\sectionTitle{精通软件}{\faCode}
	%%======================================================================
	%\begin{itemize}
	%  \item \Matlab
	% 
	%  \item Wolfram \Mathematica
	%  
	%  \item \Ansys Products
	%  
	%  \item \LaTeX
	%  
	%  \item \Microsoft Office
	%\end{itemize}
	
	%======================================================================
	\sectionTitle{个人项目}{\faCode}
	%======================================================================
	\begin{itemize}
		\begin{spacing}{1.6}
			\setlength{\itemsep}{12pt}
			\item 
			\link{https://github.com/sq43793911/Masterarbeit_public}{\texttt{\textbf{\underline{{\large 硕士论文(Masterarbeit)}}}}}: (\textit{Matlab\textregistered}, \textit{Mathematica\textregistered}, \textit{Ansys\textregistered}, \textit{Solidworks\textregistered}, \textit{\LaTeX},\textit{FEM}, \textit{PULSE LabShop}, \textit{ME'scopeVES})\\
			\link{https://github.com/sq43793911/Masterarbeit_public}{\githubSymbol $ \Rightarrow $ https://github.com/sq43793911/Masterarbeit\_public }\\[1mm]
			{\large 	\textbf{	基于偏心误差影响的HSC铣削刀柄的实验和模拟模态分析(Experimentelle und simulative Modalanalyse eines Werkzeugschaftes beim HSC-Fräsen unter Einfluss eines Exzentrizitätsfehlers)}。对HSC铣削刀柄的弯曲振动进行了模态分析,并分析了偏心误差对于结构固有频率的影响。仿真模型使用\Matlab 进行构建,并与\Ansys 结果进行比较分析。使用Solidworks进行实验用零件的绘制,使用PULSE LabShop和ME'scopeVES进行实验的测量和分析。}
			
			\item 
			\link{https://github.com/sq43793911/Projektarbeit_public}{\texttt{\textbf{\underline{{\large 项目论文(Projektarbeit)}}}}}:(\textit{Matlab\textregistered}, \textit{Mathematica\textregistered}, \textit{Ansys\textregistered}, \textit{\LaTeX},\textit{FEM})\\[1mm]
			\link{https://github.com/sq43793911/Projektarbeit_public}{\githubSymbol $ \Rightarrow $ https://github.com/sq43793911/Projektarbeit\_public }\\[1mm]
			{\large 	\textbf{基于有限元方法的模态分析(Modalanalyse mit Hilfe der Finite-Elemente-Methode)。}使用\Matlab 和 \Ansys 对杆件的轴向和弯曲振动,以及板件的弯曲振动进行了模态分析,并生成模态振型。}
			
			\item 
			\link{https://github.com/sq43793911/Intelligent-Algorithms}{\texttt{\textbf{\underline{{\large 智能算法研究项目}}}}}:(\textit{Matlab\textregistered})\\[1mm]
			\link{https://github.com/sq43793911/Intelligent-Algorithms}{\githubSymbol $ \Rightarrow $ https://github.com/sq43793911/Intelligent-Algorithms }\\[1mm]
			{\large 	$ \mathbf{a)}  $使用了4种常见的最优化智能算法(遗传、蚁群、免疫、禁忌表算法)对旅行商问题进行分析,比较了各类算法的优缺点以及应用价值。\\
				$ \mathbf{b)}  $使用遗传算法对工厂订货仓储成本管理进行了最优化分析。\\
				$ \mathbf{c)}  $基于\Matlab 的分类学习工具箱,研究并实践了数据挖掘中常用的kNN(k最邻近分类算法)和聚类分析的算法。}
			
			\item \link{https://github.com/sq43793911/Balance_Car}{\texttt{\textbf{\underline{{\large 二轮平衡车}}}}}:(\textit{Freescale\textregistered CodeWarrior}, \textit{Freescale\textregistered S12X}, \textit{C}, \textit{CCD})\\[1mm]
			\link{https://github.com/sq43793911/Balance_Car}{\githubSymbol $ \Rightarrow $ https://github.com/sq43793911/Balance\_Car }\\[1mm]
			{\large 	基于飞思卡尔智能车大赛要求设计并制作的两轮平衡车,具有赛道自动识别和控制功能。控制系统基于Freescale\textregistered S12X微控制器,使用CCD摄像头进行赛道识别。}
			
			\item \link{https://github.com/sq43793911/Construction_Analysis}{\texttt{\textbf{\underline{{\large 常见弹性振动系统分析}}}}}:(\textit{Matlab\textregistered})\\[1mm]
			\link{https://github.com/sq43793911/Construction_Analysis}{\githubSymbol $ \Rightarrow $ https://github.com/sq43793911/Construction\_Analysis }\\[1mm]
			{\large 	分析了无阻尼振动,有阻尼振动,Maxwell模型的时域特性,受激励之后的响应特性,并生成相应图像。}
			
			\item \link{https://github.com/sq43793911/resume}{\texttt{\textbf{\underline{{\large 个人简历:}}}}}(\textit{\LaTeX})\\[1mm]
			\link{https://github.com/sq43793911/resume}{\githubSymbol $ \Rightarrow $ https://github.com/sq43793911/resume }\\[1mm]
			{\large  使用\LaTeX 进行了这个简历的制作,学习了\LaTeX 的基本功能和使用方法。}
			
		\end{spacing}
		
	\end{itemize}
	
	\clearpage
	
\end{document}
