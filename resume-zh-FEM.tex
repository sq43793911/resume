%%
%% Copyright (c) 2019-2020 Qian Sun <sqandzxy@gmail.com>
%%
%%

% Chinese version
\documentclass[zh]{resume}
%\pagestyle{fancy}


\name{谦}{孙}

\keywords{Matlab, Mathematica, Programming, Construction, Material, Modalanalyse}

% \tagline{\icon{\faBinoculars}} <position-to-look-for>}
% \tagline{<current-position>}

\photo{3cm}{Bild02}

\profile{
  \mobile{136-5168-6379}
  \email{Sun.Qian-Work@outlook.com}
  \github{sq43793911} \\
  \university{弗赖贝格工业大学(Techinische Universität Bergakademie Freiberg)}
  \degree{理学 \textbullet 硕士}
  \birthday{1991-09-21}
  \home{中国 \textbullet 山东}
  % Custom information:
  % \icontext{<icon>}{<text>}
  % \iconlink{<icon>}{<link>}{<text>}
}


\begin{document}
\makeheader

%======================================================================
% Summary & Objectives
%======================================================================

	{\onehalfspacing\hspace{2em}%
汽车制造(材料与部件) 研究生,有扎实的车辆工程学、力学、材料学、振动学和嵌入式系统的基础,擅长结构建模、有限元分析、模态分析和嵌入式系统设计,熟练掌握 Matlab 和Simulik的编程及建模,以及Ansys ,Solidworks 等常用建模仿真软件,热衷计算机和网络技术,熟练运用Linux,Git,\LaTeX , Office等常用平台和工具。
拥有长达4年多的德国留学经历,具有优秀的德语沟通交流能力,硕士在校成绩优秀,积极实践科学探索精神,并积极参与多个项目和比赛。
\par}



%======================================================================
\sectionTitle{教育背景}{\faGraduationCap}
%======================================================================
	%\begin{spacing}{1.5}
	\begin{educations}
		
		
		\education%
		{2014.07}%
		[2010.09]%
		{聊城大学}%
		{\textbf{车辆工程}\textbullet 结构设计和嵌入式设计方向 }%
		{\textbf{核心课程:}机械设计,汽车理论,汽车设计,嵌入式系统设计 \textbullet 综合平均分83}%
		{\textbf{学士}}
		
		\separator{0.5ex}
		
		\education%
		{2020.07}%
		[2016.10]%
		{弗赖贝格工业大学}%
		{\textbf{汽车制造(材料与部件)} \textbullet  轻量化、有限元分析和模态分析方向}%
		{\textbf{核心课程:}结构仿真与建模,工程振动学,汽车轻量化,材料技术 \textbullet 综合平均分2.1(满分1.0)}%
		{\textbf{硕士}}
		
			
	\end{educations}
%\end{spacing}


%%======================================================================
%\sectionTitle{专业技能}{\faCogs}
%%======================================================================
%\begin{itemize}
%	\setlength{\itemsep}{5pt}
%  \item \textit{\underline{{\large 车辆工程}}}:\quad {\large \textbf{汽车轻量化技术},\quad \textbf{新能源汽车技术},\quad 汽车构造和理论}
%  \item \textit{\underline{{\large 材料科学}}}: \quad {\large \textbf{金属高温热处理技术},\quad \textbf{摩擦和磨损分析},\quad 金属腐蚀分析和腐蚀防护,仿生材料}
%  \item \textit{\underline{{\large 力学和运动学}}}: \quad {\large \textbf{模态分析},\quad 结构分析和仿真,振动系统分析和仿真}
%  \item \textit{\underline{{\large 计算机学}}}: \quad {\large \textbf{MCU系统的设计和调试}, \quad 控制系统的设计和调试}
%  \item \textit{\underline{{\large 其他}}}: \quad {\large \textbf{最优化设计},\quad 神经网络,大数据分析}
%\end{itemize}
%\bigskip
%%======================================================================
%\sectionTitle{精通软件}{\faCode}
%%======================================================================
%\begin{itemize}
%  \item \Matlab
% 
%  \item Wolfram \Mathematica
%  
%  \item \Ansys Products
%  
%  \item \LaTeX
%  
%  \item \Microsoft Office
%\end{itemize}

%======================================================================
\sectionTitle{个人项目}{\faCode}
%======================================================================
\begin{itemize}
	%\begin{spacing}{1.5}
	%\setlength{\itemsep}{12pt}
	
	\item 
	\link{https://github.com/sq43793911/Masterarbeit_public}{\texttt{\textbf{硕士论文(Masterarbeit)}}} %(\textit{Matlab}, \textit{Mathematica}, \textit{Ansys}, \textit{Solidworks}, \textit{\LaTeX},\textit{FEM}, \textit{PULSE LabShop}, \textit{ME'scopeVES})\\
	\link{https://github.com/sq43793911/Masterarbeit_public}{\githubSymbol https://github.com/sq43793911/Masterarbeit\_public }\\
	\textbf{高速切削加工中基于偏心误差影响的铣削刀柄的实验和模拟模态分析(Experimentelle und simulative Modalanalyse eines Werkzeugs\-chaftes beim HSC-Fräsen unter Einfluss eines Exzentrizitätsfehlers)}。对高速切削加工中铣削刀柄的弯曲振动进行了模态分析,并分析了偏心误差对于结构固有频率的影响。仿真模型使用Matlab 进行构建,并与Ansys 结果进行比较分析。使用Solidworks进行实验用零件的绘制,使用PULSE LabShop和ME'scopeVES进行实验的测量和分析。\\
	通过学习\textbf{非线性有限元}的基本方法,提出了对于高速运动物体的模态分析的行之有效的方法,独立完成了非线性有限元的模态分析的建模,并学习了模态分析的实验研究方法和常用实验软件的使用方法。
	
	\item 
	\link{https://github.com/sq43793911/Projektarbeit_public}{\texttt{\textbf{项目论文(Projektarbeit)}}}: %(\textit{Matlab}, \textit{Mathematica}, \textit{Ansys}, \textit{\LaTeX},\textit{FEM})\\[1mm]
	\link{https://github.com/sq43793911/Projektarbeit_public}{\githubSymbol https://github.com/sq43793911/Projektarbeit\_public }\\
	\textbf{基于有限元方法的模态分析(Modalanalyse mit Hilfe der Finite-Elemente-Methode)。}使用Matlab 和 Ansys 对杆件的轴向和弯曲振动,以及板件的弯曲振动进行了模态分析,并生成模态振型。\\
	通过该项目学习了有限元的数学建模方法和模态分析的方法,以及Ansys的使用方法。
	
	
	\item \link{https://github.com/sq43793911/Balance_Car}{\texttt{\textbf{嵌入式控制系统研究}}}:%(\textit{Freescale CodeWarrior}, \textit{Freescale S12X}, \textit{Freescale K60},\textit{C++}, \textit{CCD})\\[1mm]
	\link{https://github.com/sq43793911/Balance_Car}{\githubSymbol https://github.com/sq43793911/Balance\_Car }\\
	$\mathbf{a)}$ \textbf{飞思卡尔智能车大赛:}基于比赛要求设计并制作的两轮平衡车,具有赛道自动识别和控制功能。控制系统基于Freescale S12X微控制器,使用电磁感应原理进行赛道识别。学习了嵌入式系统设计的基本原理和常用控制算法的实现方法,并对微控制器的实际应用有了更深入的了解\\
	$\mathbf{b)}$ \textbf{学士论文:}使用Freescale K60作为控制核心,CCD摄像头作为采集设备,基于上一项研究,重新设计和制作的二轮平衡车,能够通过分析摄像头的信息进行路线或赛道的识别。学习并深入研究了32位控制器的工作原理与使用方法,以及CCD摄像头的图像采集和分析方法。
	
	\item 
	\link{https://github.com/sq43793911/Intelligent-Algorithms}{\texttt{\textbf{智能算法研究项目}}}:%(\textit{Matlab})\\[1mm]
	\link{https://github.com/sq43793911/Intelligent-Algorithms}{\githubSymbol https://github.com/sq43793911/Intelligent-Algorithms }\\
	$ \mathbf{a)}  $ \textbf{Matlab算法建模:}使用Matlab对4种常见的最优化智能算法(遗传、蚁群、免疫、禁忌表算法)进行建模,并对旅行商问题分别求解,学习了最优化智能算法的基本原理和数学建模方法,比较了各类算法的优缺点以及应用价值,方便以后进行相应算法的选择。\\
	$ \mathbf{b)}  $ \textbf{Matlab遗传算法建模:}基于经济批量模型理论,使用遗传算法对工厂订货仓储成本管理进行了最优化分析,并独立完成了整个系统的理论分析和数学建模,以及Matlab 全部代码的编写。\\
	$ \mathbf{c)}  $ \textbf{深度学习: }基于Matlab 的深度学习工具箱,研究并实践了数据挖掘中常用的kNN(k最邻近分类算法)和聚类分析的算法,学习了神经网络的基本原理和方法。
	
	
	
%	\item \link{https://github.com/sq43793911/Construction_Analysis}{\texttt{\textbf{\underline{ 常见弹性振动系统分析}}}}:(\textit{Matlab})\\[1mm]
%	\link{https://github.com/sq43793911/Construction_Analysis}{\githubSymbol $ \Rightarrow $ https://github.com/sq43793911/Construction\_Analysis }\\[1mm]
%    分析了无阻尼振动,有阻尼振动,Maxwell模型的时域特性,受激励之后的响应特性,并生成相应图像。
	
%	\item \link{https://github.com/sq43793911/resume}{\texttt{\textbf{个人简历:}}}%(\textit{\LaTeX})
%	\link{https://github.com/sq43793911/resume}{\githubSymbol https://github.com/sq43793911/resume }\\
%	使用\LaTeX 进行了这个简历的制作,学习了\LaTeX 的基本功能和使用方法。
	
%\end{spacing}
  		
\end{itemize}


%======================================================================
\sectionTitle{实习经历}{\faBriefcase}
%======================================================================
%\begin{spacing}{1.5}
\begin{experiences}
	\experience%
	[2016.08]%
	{2015.09}%
	{\textbf{售后服务} @ 山东鲁沪汽车销售有限公司}%
	[ 参与售后故障排查、修理以及技术分析,并负责相应车辆售后信息的处理、录入、归档,提高了对汽车的各个系统的了解和认识,熟练使用了Word,Excel等办公软件,并熟悉了汽车售后的基本流程和故障排查方法。
	]
\end{experiences}
%\end{spacing}


%======================================================================
\sectionTitle{技能和语言}{\faWrench}
%======================================================================
%\begin{spacing}{1.5}
\begin{competences}
	\comptence{专业领域}{%
		\textbf{模态分析},\quad \textbf{汽车轻量化},\quad \textbf{新能源汽车},\quad \textbf{有限元分析}, \quad MCU系统设计
	}
	\comptence{编程}{%
		\textbf{Matlab}, \quad Mathematica, \quad C++
	}
	\comptence{工具}{%
		\textbf{Ansys}, \quad \textbf{Solidworks}, \quad Github, \quad \textbf{\LaTeX}, \quad Microsoft \textbf{Office}, \quad Freescale \textbf{CodeWarrior}
	}
	
	\comptence{\icon{\faLanguage} 语言}{
		\textbf{德语} --- 精通;  \quad  \textbf{英语} --- 熟练(CET-4)
	}
\end{competences}
%\end{spacing}

%======================================================================
% Awards / Scholarships / Certificates
\sectionTitle{获奖及证书}{\faHeart}
%======================================================================
%\begin{spacing}{1.5}
\begin{entries}
	\entry{2013.06}%
	{第一届山东省大学生汽车知识竞赛 \textbullet 第四名}
	\entry{2013.08}%
	{第八届飞思卡尔智能车大赛山东赛区 \textbullet 优胜奖}
	\entry{2015.11}%
	{Test-Daf 德福语言证书 \textbullet 16分(满分20)}
	\entry{2016.03}%
	{APS留德人员审核部 \textbullet 学历审核证书}
	%	\entry{2011.09}%
	%	{全国计算机等级考试 \textbullet 四级网络工程师}
\end{entries}
%\end{spacing}

\clearpage

\end{document}
