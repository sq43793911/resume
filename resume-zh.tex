%%
%% Copyright (c) 2019-2020 Qian Sun <sqandzxy@gmail.com>
%%
%%

% Chinese version
\documentclass[zh]{resume}

% File information shown at the footer of the last page
%\fileinfo{%
%  \faCopyright{} 2018--2019 Weitian LI,
%  \creativecommons{by}{4.0},
%  \githublink{liweitianux}{resume},
%  \faEdit{} \today
%}
\newcommand{\Matlab}{\textsc{Matlab}\textsuperscript{\textregistered} }
\newcommand{\Ansys}{\textsc{Ansys}\textsuperscript{\textregistered} }
\newcommand{\Mathematica}{\textsc{Mathematica}\textsuperscript{\textregistered} }
\newcommand{\Microsoft}{\textsc{Microsoft}\textsuperscript{\textregistered} }

\name{谦}{孙}

\keywords{Matlab, Mathematica, Programming, Construction, Material}

% \tagline{\icon{\faBinoculars}} <position-to-look-for>}
% \tagline{<current-position>}

% \photo{<height>}{<filename>}

\profile{
  \mobile{+49 0152-5754-1682}
  \email{sqandzxy@gmail.com}
  %\github{sq43793911} \\
  \university{Techinische Universität Bergakademie Freiberg} \\
  \degree{工学 \textbullet 硕士}
  \birthday{1991-09-21}
  \home{Deutschland \textbullet Freiberg}
  % Custom information:
  % \icontext{<icon>}{<text>}
  % \iconlink{<icon>}{<link>}{<text>}
}


\begin{document}
\makeheader

%======================================================================
% Summary & Objectives
%======================================================================
{\onehalfspacing\hspace{2em}%
车辆工程(材料和结构方向) 在读研究生,有扎实的物理、数学与材料学基础,
擅长结构建模与分析,热衷计算机和网络技术,熟练运用Linux,SSH,Git等常用平台和工具。
熟悉常用的科学计算软件,熟练掌握 Matlab和Mathematica语言编程,熟练掌握Ansys等有限元分析软件。
积极实践自由开源精神,并积极参多个项目和比赛。
\par}

%======================================================================
\sectionTitle{技能和语言}{\faWrench}
%======================================================================
\begin{competences}
  \comptence{专业领域}{%
    汽车新能源方向,汽车轻量化方向,材料处理和结构方向,结构振动分析和优化方向, MCU系统设计
  }
  \comptence{编程}{%
    \textbf{\Matlab}, \quad \Mathematica, \quad R, \quad Python, \quad \textbf{C}
  }
  \comptence{工具}{%
    \Ansys Products, \quad SSH, \quad Git, \quad \textbf{\LaTeX}, \quad \Microsoft \textbf{Office}, \quad Freescale\textregistered \textbf{CodeWarrior}, \quad IAR\textregistered Embedded Workbench
  }
%  \comptence{数据分析}{%
%    R, Pandas; Matplotlib, ggplot2; Keras, Scikit-learn
%  }
%  \comptence{网站开发}{%
%    Flask, JavaScript, jQuery, Bootstrap
%  }
  \comptence{\icon{\faLanguage} 语言}{
    \textbf{德语} --- \textbf{TestDaf}(4 \texttimes4) ;    \textbf{英语} --- 读写(流利),听说(日常交流)
  }
\end{competences}

%======================================================================
\sectionTitle{教育背景}{\faGraduationCap}
%======================================================================
\begin{educations}
  \education%
    {2016.10}%
    {Technische Universität Bergakademie Freiberg}%
    {Werkstoffwissenschaft und Werkstofftechnologie}%
    {Fahrzeugbau: Werkstoffe und Komponenten 平均分:2.4(81)}%
    {硕士}

  \separator{0.5ex}
  \education%
    {2010.09}%
    [2014.06]%
    {聊城大学}%
    {机械与汽车工程学院}%
    {车辆工程 \quad 平均分:2.3(83)}%
    {学士}
\end{educations}

%======================================================================
\sectionTitle{专业技能}{\faCogs}
%======================================================================
\begin{itemize}
  \item 车辆工程:汽车构造和理论;新能源汽车技术,\textbf{汽车轻量化技术}
  \item 材料科学:金属材料成型和处理,\textbf{金属铸造和高温热处理技术},仿生材料
  \item 力学和运动学:结构分析和仿真,振动分析和仿真,\textbf{模态分析}
  \item 计算机学:\textbf{MCU系统的设计和调试},控制系统的设计和调试
  \item 其他:\textbf{最优化设计},神经网络,大数据分析
\end{itemize}

%%======================================================================
%\sectionTitle{精通软件}{\faCode}
%%======================================================================
%\begin{itemize}
%  \item \Matlab
% 
%  \item Wolfram \Mathematica
%  
%  \item \Ansys Products
%  
%  \item \LaTeX
%  
%  \item \Microsoft Office
%\end{itemize}

%======================================================================
\sectionTitle{个人项目}{\faCode}
%======================================================================
\begin{itemize}
	\item \link{https://github.com/sq43793911/Balance_Car}{\texttt{二轮平衡车}}:(Freescale\textregistered CodeWarrior, Freescale\textregistered S12X, CCD)基于飞思卡尔智能车大赛要求设计并制作的两轮平衡车,具有赛道自动识别和控制功能。
	
	\item \link{https://github.com/sq43793911/TSP}{\texttt{TSP问题的多种算法分析}}:(\Matlab)使用了4种常见的寻优算法(遗传、蚁群、免疫、禁忌表算法)对TSP问题进行分析,比较了各类算法的优缺点以及应用价值。
	
	\item \link{https://github.com/sq43793911/Construction_Analysis}{\texttt{常见弹性振动系统分析}}:(\Matlab)分析了无阻尼振动,有阻尼振动,Maxwell模型的时域特性,受激励之后的响应特性,并生成相应图像。
	
	\item \link{https://github.com/sq43793911/Projektarbeit_public}{\texttt{Projektarbeit}}: (\Matlab, \Mathematica, \Ansys, \LaTeX)
	      Modalanalyse mit Hilfe der Finite-Elemente-Methode (Modalanalysis with the Finite-Element-Method). 使用\Matlab 和 \Ansys 对杆件的轴向和弯曲振动,以及板件的弯曲振动进行了模态分析,并生成模态振型。
  		
\end{itemize}

\clearpage

%======================================================================
% Internships
\sectionTitle{实习经历}{\faBriefcase}
%======================================================================
%\begin{experiences}
%	\experience
%	[2018.04]%
%	{2018.08}%
%	{数据工程师 @ 上海领脉网络科技(初创公司)}%
%	[\begin{itemize}
%		\item 从 Amazon 网页搜索并挖取商品与广告信息
%		(Python, Requests, BeautifulSoup)
%		\item 配置 Airflow 服务器和数据库等基础设施,定期从 Amazon 获取
%		产品销售与广告投放等数据
%		\item 开发网站(Flask, jQuery),帮助客户优化 Amazon 广告投放
%	\end{itemize}]%
%	
%	\separator{0.5em}
%	\experience
%	[2013.07]%
%	{2013.09}%
%	{网站开发 @ 97 随访(初创公司)}%
%	[\begin{itemize}
%		\item 后端开发(Django),完成用户注册、数据存储和搜索等功能
%		\item 前端开发(jQuery, AJAX),对患者各项指标随时间的变化进行可视化
%	\end{itemize}]%
%\end{experiences}

%%======================================================================
%\sectionTitle{实习经历}{\faBriefcase}
%%======================================================================
%\begin{experiences}
%  \experience%
%    [2018.04]%
%    {2018.08}%
%    {数据工程师 @ 上海领脉网络科技(初创公司)}%
%    [\begin{itemize}
%      \item 从 Amazon 网页搜索并挖取商品与广告信息
%        (Python, Requests, BeautifulSoup)
%      \item 配置 Airflow 服务器和数据库等基础设施,
%        定期从 Amazon 获取产品销售与广告投放等数据
%      \item 开发网站(Flask, jQuery),帮助客户优化 Amazon 广告投放
%    \end{itemize}]
%
%  \separator{0.5ex}
%  \experience%
%    [2013.07]%
%    {2013.09}%
%    {网站开发 @ 97 随访(初创公司)}%
%    [\begin{itemize}
%      \item 后端开发(Django),完成用户注册、数据存储和搜索等功能
%      \item 前端开发(jQuery, AJAX),对患者各项指标随时间的变化进行可视化
%    \end{itemize}]
%\end{experiences}

\end{document}
