%%
%% Copyright (c) 2019-2020 Qian Sun <sqandzxy@gmail.com>
%%
%%

% Chinese version
\documentclass[zh]{resume}
\pagestyle{fancy}

\newcommand{\Matlab}{\textsc{Matlab}\textregistered }
\newcommand{\Ansys}{\textsc{Ansys}\textregistered }
\newcommand{\Mathematica}{\textsc{Mathematica}\textregistered }
\newcommand{\Microsoft}{\textsc{Microsoft}\textregistered }
\newcommand{\Solidworks}{\textsc{Solidworks}\textregistered }

\name{谦}{孙}

\keywords{Matlab, Mathematica, Programming, Construction, Material, Modalanalyse}

% \tagline{\icon{\faBinoculars}} <position-to-look-for>}
% \tagline{<current-position>}

% \photo{<height>}{<filename>}

\profile{
  \mobile{+49 015257541682}
  \email{sqandzxy@gmail.com}
  \github{sq43793911} \\
  \university{弗莱贝格工业大学(Techinische Universität Bergakademie Freiberg)} \\
  \degree{理学 \textbullet 硕士}
  \birthday{1991-09-21}
  \home{德国 \textbullet 弗莱贝格(Freiberg)}
  % Custom information:
  % \icontext{<icon>}{<text>}
  % \iconlink{<icon>}{<link>}{<text>}
}


\begin{document}
\makeheader

%======================================================================
% Summary & Objectives
%======================================================================
%{\onehalfspacing\hspace{2em}%
%车辆工程(材料和结构方向) 研究生,有扎实的物理、数学与材料学基础,
%擅长结构建模与分析,热衷计算机和网络技术,熟练运用Linux,Git,\LaTeX , Office等常用平台和工具。
%熟悉常用的科学计算软件,熟练掌握 \Matlab 语言编程,熟练掌握\Ansys ,\Solidworks 等常用建模仿真软件。
%积极实践科学探索精神,并积极参与多个项目和比赛。
%\par}

%======================================================================
\sectionTitle{技能和语言}{\faWrench}
%======================================================================
\begin{spacing}{1.4}
\begin{competences}
  \comptence{专业领域}{%
    {\large \textbf{新能源汽车},\quad \textbf{汽车轻量化},\quad \textbf{模态分析},\quad \textbf{有限元分析}, \quad MCU系统设计}
  }
  \comptence{编程}{%
    {\large \textbf{Matlab\textregistered}, \quad Mathematica\textregistered, \quad R\textregistered, \quad Python\textregistered, \quad C++\textregistered}
  }
  \comptence{工具}{%
    {\large \textbf{Ansys\textregistered}Products, \quad \textbf{Solidworks\textregistered}, \quad Git, \quad \textbf{\LaTeX}, \quad \Microsoft \textbf{Office}, \quad Freescale\textregistered \textbf{CodeWarrior}}
  }

  \comptence{\icon{\faLanguage} 语言}{
    {\large \textbf{德语} --- C1;  \quad  \textbf{英语} --- B1}
  }
\end{competences}
\end{spacing}

%======================================================================
\sectionTitle{教育背景}{\faGraduationCap}
%======================================================================
\begin{spacing}{1.4}
\begin{educations}
	
	\education%
	{2014.07}%
	[2010.09]%
	{聊城大学}%
	{机械与汽车工程学院}%
	{车辆工程}%
	{学士}

  \separator{0.5ex}
  
  \education%
  {2020.07}%
  [2016.10]%
  {弗莱贝格工业大学(Technische Universität Bergakademie Freiberg)}%
  {材料科学与技术系(Werkstoffwissenschaft und Werkstofftechnologie)}%
  {汽车制造:材料和结构(Fahrzeugbau: Werkstoffe und Komponenten)}%
  {硕士}

\end{educations}
\end{spacing}

%======================================================================
\sectionTitle{专业技能}{\faCogs}
%======================================================================
\begin{itemize}
	\setlength{\itemsep}{5pt}
  \item \textit{\underline{{\large 车辆工程}}}:\quad {\large \textbf{汽车轻量化技术},\quad \textbf{新能源汽车技术},\quad 汽车构造和理论}
  \item \textit{\underline{{\large 材料科学}}}: \quad {\large \textbf{金属高温热处理技术},\quad \textbf{摩擦和磨损分析},\quad 金属腐蚀分析和腐蚀防护,仿生材料}
  \item \textit{\underline{{\large 力学和运动学}}}: \quad {\large \textbf{模态分析},\quad 结构分析和仿真,振动系统分析和仿真}
  \item \textit{\underline{{\large 计算机学}}}: \quad {\large \textbf{MCU系统的设计和调试}, \quad 控制系统的设计和调试}
  \item \textit{\underline{{\large 其他}}}: \quad {\large \textbf{最优化设计},\quad 神经网络,大数据分析}
\end{itemize}

%%======================================================================
%\sectionTitle{精通软件}{\faCode}
%%======================================================================
%\begin{itemize}
%  \item \Matlab
% 
%  \item Wolfram \Mathematica
%  
%  \item \Ansys Products
%  
%  \item \LaTeX
%  
%  \item \Microsoft Office
%\end{itemize}

%======================================================================
\sectionTitle{个人项目}{\faCode}
%======================================================================
\begin{itemize}
	\begin{spacing}{1.4}
	\setlength{\itemsep}{12pt}
	\item 
	\link{https://github.com/sq43793911/Masterarbeit_public}{\texttt{\textbf{\underline{{\large 硕士论文(Masterarbeit)}}}}}: (\textit{Matlab\textregistered}, \textit{Mathematica\textregistered}, \textit{Ansys\textregistered}, \textit{Solidworks\textregistered}, \textit{\LaTeX},\textit{FEM}, \textit{PULSE LabShop}, \textit{ME'scopeVES})\\
	\link{https://github.com/sq43793911/Masterarbeit_public}{\githubSymbol $ \Rightarrow $ https://github.com/sq43793911/Masterarbeit\_public }\\[1mm]
{\large 	\textbf{	基于偏心误差影响的HSC铣削刀柄的实验和模拟模态分析(Experimentelle und simulative Modalanalyse eines Werkzeugschaftes beim HSC-Fräsen unter Einfluss eines Exzentrizitätsfehlers)}。对HSC铣削刀柄的弯曲振动进行了模态分析,并分析了偏心误差对于结构固有频率的影响。仿真模型使用\Matlab 进行构建,并与\Ansys 结果进行比较分析。使用Solidworks进行实验用零件的绘制,使用PULSE LabShop和ME'scopeVES进行实验的测量和分析。}
	
	\item 
	\link{https://github.com/sq43793911/Projektarbeit_public}{\texttt{\textbf{\underline{{\large 项目论文(Projektarbeit)}}}}}:(\textit{Matlab\textregistered}, \textit{Mathematica\textregistered}, \textit{Ansys\textregistered}, \textit{\LaTeX},\textit{FEM})\\[1mm]
	\link{https://github.com/sq43793911/Projektarbeit_public}{\githubSymbol $ \Rightarrow $ https://github.com/sq43793911/Projektarbeit\_public }\\[1mm]
{\large 	\textbf{基于有限元方法的模态分析(Modalanalyse mit Hilfe der Finite-Elemente-Methode)。}使用\Matlab 和 \Ansys 对杆件的轴向和弯曲振动,以及板件的弯曲振动进行了模态分析,并生成模态振型。}
	
	\item 
	\link{https://github.com/sq43793911/Intelligent-Algorithms}{\texttt{\textbf{\underline{{\large 智能算法研究项目}}}}}:(\textit{Matlab\textregistered})\\[1mm]
	\link{https://github.com/sq43793911/Intelligent-Algorithms}{\githubSymbol $ \Rightarrow $ https://github.com/sq43793911/Intelligent-Algorithms }\\[1mm]
{\large 	$ \mathbf{a)}  $使用了4种常见的最优化智能算法(遗传、蚁群、免疫、禁忌表算法)对旅行商问题进行分析,比较了各类算法的优缺点以及应用价值。\\
	$ \mathbf{b)}  $使用遗传算法对工厂订货仓储成本管理进行了最优化分析。\\
	$ \mathbf{c)}  $基于\Matlab 的分类学习工具箱,研究并实践了数据挖掘中常用的kNN(k最邻近分类算法)和聚类分析的算法。}
	
	\item \link{https://github.com/sq43793911/Balance_Car}{\texttt{\textbf{\underline{{\large 二轮平衡车}}}}}:(\textit{Freescale\textregistered CodeWarrior}, \textit{Freescale\textregistered S12X}, \textit{C}, \textit{CCD})\\[1mm]
	\link{https://github.com/sq43793911/Balance_Car}{\githubSymbol $ \Rightarrow $ https://github.com/sq43793911/Balance\_Car }\\[1mm]
{\large 	基于飞思卡尔智能车大赛要求设计并制作的两轮平衡车,具有赛道自动识别和控制功能。控制系统基于Freescale\textregistered S12X微控制器,使用CCD摄像头进行赛道识别。}
	
	\item \link{https://github.com/sq43793911/Construction_Analysis}{\texttt{\textbf{\underline{{\large 常见弹性振动系统分析}}}}}:(\textit{Matlab\textregistered})\\[1mm]
	\link{https://github.com/sq43793911/Construction_Analysis}{\githubSymbol $ \Rightarrow $ https://github.com/sq43793911/Construction\_Analysis }\\[1mm]
{\large 	分析了无阻尼振动,有阻尼振动,Maxwell模型的时域特性,受激励之后的响应特性,并生成相应图像。}
	
	\item \link{https://github.com/sq43793911/resume}{\texttt{\textbf{\underline{{\large 个人简历:}}}}}(\textit{\LaTeX})\\[1mm]
	\link{https://github.com/sq43793911/resume}{\githubSymbol $ \Rightarrow $ https://github.com/sq43793911/resume }\\[1mm]
	{\large  使用\LaTeX 进行了这个简历的制作,学习了\LaTeX 的基本功能和使用方法。}
	
\end{spacing}
  		
\end{itemize}

\clearpage

\end{document}
